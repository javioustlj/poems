% !Mode:: "TeX:UTF-8"
% poems.tex
% 夏田墨诗词集

\documentclass[twoside]{book}



%========基本必备的宏包========%

\usepackage{calc} % 用来在源文件中做简单数学计算的宏包
\usepackage{zhnumber} % 用来将阿拉伯数字转换为中文数字

\usepackage{geometry} % 用来设置页面的宏包
\usepackage{tocloft} % 用来设置目录

\usepackage{titlesec} % 用于设置标题
\usepackage{fancyhdr} % 用来设置页眉页脚
\usepackage{perpage} % 用于每页计数的宏包
\usepackage{multicol} % 用来多栏排版的宏包
\usepackage{verse} % 用于排版诗歌
\usepackage{indentfirst} % 控制首行缩进的宏包
\usepackage{hyperref} % 用于创建超链接
\usepackage{framed} % 带边框的文本或段落
\usepackage{xcolor} % 用来设置颜色的宏包

% \usepackage{xeCJK} % 用于中文字体的宏包
\usepackage{float} % 用来设置浮动字体的宏包
\usepackage{fontspec} % 用于处理字体
\usepackage{xpinyin} % 用来给汉字注音
\usepackage{moresize} % 提供更多字体大小控制的宏包
\usepackage{tocbibind} %把目录本身加入目录

%=========页面设置=========%
\geometry{
    a4paper, %a4paper size 297:210 mm
    bindingoffset=10mm, %装订线
    top=35mm,
    bottom=30mm,
    inner=10mm, % 左边距, left 是在单面文档中使用, inner是双面文档使用
    outer=10mm, % 右边距,right 是在单面文档中使用,outer是在双面文档使用
}

%=========中文设置=========%
\usepackage[UTF8, heading=true, scheme=chinese, fontset=none]{ctex}
\ctexset{fontset=fandol, space=true}
%================字体================%
%中文環境
\xeCJKsetup{PunctStyle=plain}
\setCJKmainfont{LXGW WenKai}

%=========颜色设置=========%^M
\usepackage{eso-pic}
\definecolor{lightgreen}{HTML}{CCE8CF}
\AddToShipoutPictureBG{\color{lightgreen}\rule{\paperwidth}{\paperheight}}



%=========章節標題設計=========%
%修改part
\titleformat{\part}{\huge\sffamily\centering}{}{0em}{}
%修改chapter
\titleformat{\chapter}{\LARGE\sffamily\centering}{第\,\thechapter\,章}{1em}{}
%修改section
\titleformat{\section}{\Large\sffamily\centering}{}{0em}{}
%修改subsection
\titleformat{\subsection}{\large\sffamily\centering}{}{0em}{}
%修改subsubsection
\titleformat{\subsubsection}{\normalsize\sffamily\centering}{}{0em}{}

%================目录===============%
%%===============中文=========%
\renewcommand\contentsname{目~录}
% \renewcommand\listfigurename{插图目录}
% \renewcommand\listtablename{表格目录}
% \renewcommand\bibname{参~考~文~献}
% \renewcommand\indexname{索~引}
% \renewcommand\figurename{图}
% \renewcommand\tablename{表}
% \renewcommand\partname{部分}
% \renewcommand\appendixname{附录}
\renewcommand{\today}{\number\year 年\number\month 月\number\day 日}
% \renewcommand\cfttoctitlefont{\LARGE\sffamily\centering}
\renewcommand{\cfttoctitlefont}{\hfil\LARGE\sffamily\MakeUppercase\centering}

%=======页眉页脚格式=========%
\pagestyle{fancy}
\renewcommand{\sectionmark}[1]
{\markright{第\zhnumber{\arabic{section}}节~~#1}{}}

\fancypagestyle{plain}{%
    \fancyhf{}
    \renewcommand{\headrulewidth}{0pt}
    \renewcommand{\footrulewidth}{0pt}
    \fancyhf[HR]{\ttfamily \footnotesize \rightmark }
    \fancyhf[FR]{\thepage}}
\pagestyle{plain}




%========脚注=========%
\MakePerPage{footnote}
\setlength{\skip\footins}{20pt plus 10pt}

%=======拼音========%

\xpinyinsetup{ratio={.7},hsep={.5em plus .1em}, vsep={1.1em}, multiple={\color{red}}}


%========================诗词设置==========================%
\newcommand{\poemauthor}[1]
{
    {\centering \fontsize{13}{1} #1\par}
}
\newenvironment{poem}[3]%
{
    \newpage
    \poemtitle{\fontsize{16pt}{20pt}\selectfont#1}
    \poemauthor{#2}
%    \begin{pinyinscope}
    \setlength{\stanzaskip}{3ex}
    \settowidth{\versewidth}{#3}
    \begin{verse}[\versewidth]
    \fontsize{12}{1}
}
{
    \end{verse}
%    \end{pinyinscope}
}

\hypersetup{
    pdfcreator={夏田墨}
}

\linespread{1.6}

\setcounter{tocdepth}{2}

%\pagecolor{yellow}

\title{诗词集}
\author{夏田墨}
\date{\today}


\begin{document}
\frontmatter
\maketitle  % 输出书籍名称
% \chapter*{\contentsname}
% \addcontentsline{toc}{chapter}{\contentsname} % book 类,把目录本身加入目录

\chapter{前言}
\setlength{\parskip}{0pt plus 1pt}
一直有一个想法,可以搜集整理自己喜欢的诗词歌曲。

一个能被诗词打动的人,注定不是一个冷漠的人。

该文稿采用\LaTeX 排版,并为诗词部分添加了注音,方便阅读。

\newpage
% \chapter*{\contentsname}
% \addcontentsline{toc}{chapter}{\contentsname}
\tableofcontents % 生成目录

\mainmatter

\part{古诗}
\chapter{五言诗}


\begin{poem}{春晓}{孟浩然}{春眠不觉晓}
春眠不觉晓 \\
处处闻啼鸟 \\
夜来风雨声 \\
花落知多少 \\
\end{poem}

\begin{poem}{悯农}{李绅}{锄禾日当午}
锄禾日当午 \\
汗滴禾下土 \\
谁知盘中餐 \\
粒粒皆辛苦 \\
\end{poem}


\chapter{现代诗}

\begin{poem}{三弦}{沈尹默}{旁边有一段低低土墙}
中午时候 \\
火一样的太阳 \\
没法去遮拦 \\
让他直晒着长街 \\
静悄悄少人行路 \\
只有悠悠风来 \\
吹动路旁杨树 \\!

谁家破大门里 \\
半兜子绿茸茸细草 \\
都浮若闪闪的金光 \\
旁边有一段低低土墙 \\
挡住了一个弹三弦的人 \\
却不能隔断那三弦鼓荡的声浪 \\!

门外坐着一个穿破衣裳的老年人 \\
双手抱着头 \\
他不声不响 \\!
\end{poem}

\part{歌曲}

\chapter{现代歌曲}

\begin{poem}{千百度}{许嵩}{三四更雪 风不减 吹袭一夜}
关外野店 烟火绝 客怎眠\\
寒来袖间 谁为我 添两件\\
三四更雪 风不减 吹袭一夜\\
只是可怜 瘦马未得好歇\\!

怅然入梦 梦几月 醒几年 \\
往事凄艳 用情浅 两手缘 \\
鹧鸪清怨 听得见 飞不回堂前 \\
旧楹联红褪墨残谁来揭 \\!

我寻你千百度 日出到迟暮 \\
一瓢江湖我沉浮 \\
我寻你千百度 又一岁荣枯 \\
可你从不在 灯火阑珊处 \\!

怅然入梦 梦几月 醒几年 \\
往事凄艳 用情浅 两手缘 \\
鹧鸪清怨 听得见 飞不回堂前 \\
旧楹联红褪墨残谁来揭 \\!

我寻你千百度 日出到迟暮 \\
一瓢江湖我沉浮 \\
我寻你千百度 又一岁荣枯 \\
可你从不在 灯火阑珊处 \\!

我寻你千百度 日出到迟暮 \\
一瓢江湖我沉浮 \\
我寻你千百度 又一岁荣枯 \\
你不在 灯火阑珊处 \\!
\end{poem}

\begin{poem}{庐州月}{许嵩}{如今的你又在谁的身旁}
儿时凿壁偷了谁家的光\\
宿昔不梳 一苦十年寒窗\\
如今灯下闲读 红袖添香\\
半生浮名只是虚妄\\
三月 一路烟霞 莺飞草长\\
柳絮纷飞里看见了故乡\\
不知心上的你是否还在庐阳\\!

一缕青丝一生珍藏\\
桥上的恋人入对出双\\
桥边红药叹夜太漫长\\
月也摇晃 人也彷徨\\
乌蓬里传来了一曲离殇\\!

庐州月光 洒在心上\\
月下的你不复当年模样\\
太多的伤 难诉衷肠\\
叹一句当时只道是寻常\\!

庐州月光 梨花雨凉\\
如今的你又在谁的身旁\\
家乡月光 深深烙在我心上\\
却流不出当年泪光\\!

三月 一路烟霞 莺飞草长\\
柳絮纷飞里看见了故乡\\
不知心上的你是否还在庐阳\\
一缕青丝一生珍藏\\
桥上的恋人入对出双\\
桥边红药叹夜太漫长\\
月也摇晃 人也彷徨\\
乌蓬里传来了一曲离殇\\!

庐州月光 洒在心上\\
月下的你不复当年模样\\
太多的伤 难诉衷肠\\
叹一句当时只道是寻常\\!

庐州月光 梨花雨凉\\
如今的你又在谁的身旁\\
家乡月光 深深烙在我心上\\
却流不出当年泪光\\
庐州的月光在我心上\\
太多的伤难诉衷肠\\
如今的你\\
在谁的身旁\\
我流不出当年泪光\\!

庐州月光 洒在心上\\
月下的你不复当年模样\\
太多的伤 难诉衷肠\\
叹一句当时只道是寻常\\!

庐州月光 梨花雨凉\\
如今的你又在谁的身旁\\
家乡月光 深深烙在我心上\\
却流不出当年泪光\\
\end{poem}

\begin{poem}{希望}{李宗盛}{这一首歌无关儿女情长}
养几个孩子\\
是我人生的愿望\\
我喜欢他们围绕在我身旁\\
如果这纷乱的世界让我沮丧\\
我就去看看他们眼中的光芒\\
总有一天我会越来越忙\\
还好孩子总是给我希望\\
看着他们一天一天成长\\
我真的忍不住\\
要把梦想对她讲\\
总在他们的身上看到\\
自己过去的模样\\
对自己\\
对人生\\
对未来的渴望\\
他们是我的希望\\
让我有继续的力量\\
他们是未来的希望\\
所有的孩子都一样\\
他们是未来的希望\\
但愿我能给她一个\\
最像天堂的地方\\!

依稀记得他们\\
出生时的模样\\
我和太太眼里泛着泪光\\
虽然她长得和我不是很像\\
但是朋友都说她比我漂亮\\!
毫无意外我真的越来越忙\\
还好孩子总是给我希望\\
如果能够\\
陪着他们一起成长\\
生命里就算失去一些\\
别的又怎么样\\
总在他们身上看到\\
自己过去的模样\\
对自己\\
对人生\\
对未来\\
的渴望\\
他们是我的希望\\
让我有继续的力量\\
他们是未来的希望\\
所有的孩子都一样\\
他们是未来的希望\\
但愿我能给她一个\\
最像天堂的地方\\
总在他们的身上看到\\
自己过去的模样\\
对自己\\
对人生\\
对未来\\
的渴望\\
他们是我的希望\\
让我有继续的力量\\
他们是未来的希望\\
所有的孩子都一样\\
他们是未来的希望\\
但愿我能给她一个\\
最像天堂的地方\\
虽然我难免还是会想\\
这样的歌很少人会欣赏\\
这一首歌无关儿女情长\\
只献给我家那两个\\
可爱的姑娘\\
他们在我心里\\
最柔软的地方\\
虽然我总是身在远方\\
我生命里美好的一切\\
愿与她们分享\\
\end{poem}



\backmatter
\chapter{参考资料}


\end{document}
